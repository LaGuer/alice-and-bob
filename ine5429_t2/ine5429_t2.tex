\documentclass{article}

\usepackage[brazil]{babel}
\usepackage[T1]{fontenc}
\usepackage[a4paper, margin=1.5cm]{geometry}
\usepackage[colorlinks, urlcolor=blue, citecolor=red]{hyperref}
\usepackage[utf8]{inputenc}
\usepackage{amsfonts, mathtools}

\title{\textbf{Raízes primitivas módulo $n$}}
\author{Gustavo Zambonin\thanks{\texttt{gustavo.zambonin@grad.ufsc.br} ---
todos os algoritmos utilizados podem ser encontrados
também \href{https://github.com/zambonin/ufsc-ine5429}{neste repositório}.} \\
\small {Segurança em Computação (UFSC -- INE5429)} \vspace{-5mm}}
\date{}

\begin{document}

\maketitle

\begin{itemize}

\item O escopo deste trabalho é discutir o conceito de \emph{raízes primitivas
módulo $n$}. Este construto matemático pode ser definido da seguinte forma: seja
$n$ um número inteiro; então $g$ é uma raiz primitiva módulo $n$ se, para cada
inteiro $a$ coprimo\footnote{números são coprimos, ou primos entre si, se seu
único divisor positivo em comum é o número 1. A \emph{função totiente de Euler},
$\phi(n)$, é responsável por fornecer a contagem de coprimos para um inteiro
$n$.} a $n$, existe um inteiro $k \in \mathbb{Z}^*$ tal que $g^k \equiv a \pmod
n$. Verifica-se, por exemplo, que 2 é uma raiz primitiva módulo 13:

\begin{equation*}
\begin{aligned}
2^0    &=& 1    &=& 2^0            &\equiv& 1          &=& 1  &\equiv& 1  &
\pmod{13} \\
2^1    &=& 2    &=& 2^0    \cdot 2 &\equiv& 1  \cdot 2 &=& 2  &\equiv& 2  &
\pmod{13} \\
2^2    &=& 4    &=& 2^1    \cdot 2 &\equiv& 2  \cdot 2 &=& 4  &\equiv& 4  &
\pmod{13} \\
2^3    &=& 8    &=& 2^2    \cdot 2 &\equiv& 4  \cdot 2 &=& 8  &\equiv& 8  &
\pmod{13} \\
2^4    &=& 16   &=& 2^3    \cdot 2 &\equiv& 8  \cdot 2 &=& 16 &\equiv& 3  &
\pmod{13} \\
2^5    &=& 32   &=& 2^4    \cdot 2 &\equiv& 3  \cdot 2 &=& 6  &\equiv& 6  &
\pmod{13} \\
2^6    &=& 64   &=& 2^5    \cdot 2 &\equiv& 6  \cdot 2 &=& 12 &\equiv& 12 &
\pmod{13} \\
2^7    &=& 128  &=& 2^6    \cdot 2 &\equiv& 12 \cdot 2 &=& 24 &\equiv& 11 &
\pmod{13} \\
2^8    &=& 256  &=& 2^7    \cdot 2 &\equiv& 11 \cdot 2 &=& 22 &\equiv& 9  &
\pmod{13} \\
2^9    &=& 512  &=& 2^8    \cdot 2 &\equiv& 9  \cdot 2 &=& 18 &\equiv& 5  &
\pmod{13} \\
2^{10} &=& 1024 &=& 2^9    \cdot 2 &\equiv& 5  \cdot 2 &=& 10 &\equiv& 10 &
\pmod{13} \\
2^{11} &=& 2048 &=& 2^{10} \cdot 2 &\equiv& 10 \cdot 2 &=& 20 &\equiv& 7  &
\pmod{13} \\
2^{12} &=& 4096 &=& 2^{11} \cdot 2 &\equiv& 7  \cdot 2 &=& 14 &\equiv& 1  &
\pmod{13} \\
&&&&&&&&&&& \dots
\end{aligned}
\end{equation*}
No caso de números $n$ primos, as potências de $g$ formam um ciclo que não pode
ser maior do que $n - 1$ (pelo Pequeno Teorema de Fermat). Assim, $g$ é uma raiz
primitiva módulo $n$ pois produz todos os \emph{resíduos} possíveis módulo $n$.
Nota-se que um número só possui esta característica caso a cardinalidade do
conjunto destas potências seja igual a $\phi(n)$. Como um número primo $s$ não
tem divisores a não ser 1 e ele mesmo, então todos os números no intervalo $[1,
s-1]$ são coprimos a ele, logo $\phi(s) = s - 1$.

\item A partir do conceito elaborado acima, é possível encontrar facilmente
todas as raízes primitivas de um número se apenas uma já é conhecida. Seja $g$
uma raiz primitiva módulo $n$, onde $n$ é um primo ímpar (para simplicidade da
demonstração). Então sabe-se que $g$ pode gerar todas as potências da forma $g^m
\equiv 1 \pmod n)$, para $(m = 1 \dots n - 1)$. Para estes números serem raízes
primitivas, $(a^m)^{(n - 1)/d} \equiv (a^{n - 1})^{m/d} \equiv 1 \pmod n$, então
precisa-se de $d = 1$. O código em \texttt{primitive\_root.py} explora uma
variante dessa análise que utiliza-se da escolha de números aleatórios como
`'chutes`' para a possível raiz primitiva (estes números devem estar dentro do
grupo multiplicativo $\mathbb{Z}/n\mathbb{Z}$\footnote{o conjunto de classes de
congruência relativamente primos ao número.} para que a análise seja válida).

A partir da congruência $a^{(p-1)/q} \not\equiv 1 \pmod p$ para todos os fatores
primos de $p-1$, é possível verificar se $a$ é uma raiz primitiva módulo $n$, e
construir uma lista sucessivamente. A complexidade para achar uma raiz primitiva
singular aleatoriamente funciona
\href{http://math.stackexchange.com/a/156250}{relativamente bem}, porém a
velocidade diminui exponencialmente em comparação com uma estratégia de iteração
sobre todos os valores possíveis quando é necessário construir uma lista
completa.
\begin{verbatim}
$ python
>>> import primitive_root as pr
>>> for i in [2, 3, 5, 7, 11, 13, 17, 19, 23, 29, 31, 37, 41, 43, 47]:
...     print("Primitive roots of {}: {}".format(i, pr.prim_roots(i, False)))
...
Primitive roots of 2: [1]
Primitive roots of 3: [2]
Primitive roots of 5: [2, 3]
Primitive roots of 7: [3, 5]
Primitive roots of 11: [2, 6, 7, 8]
Primitive roots of 13: [2, 6, 7, 11]
Primitive roots of 17: [3, 5, 6, 7, 10, 11, 12, 14]
Primitive roots of 19: [2, 3, 10, 13, 14, 15]
Primitive roots of 23: [5, 7, 10, 11, 14, 15, 17, 19, 20, 21]
Primitive roots of 29: [2, 3, 8, 10, 11, 14, 15, 18, 19, 21, 26, 27]
Primitive roots of 31: [3, 11, 12, 13, 17, 21, 22, 24]
Primitive roots of 37: [2, 5, 13, 15, 17, 18, 19, 20, 22, 24, 32, 35]
Primitive roots of 41: [6, 7, 11, 12, 13, 15, 17, 19, 22, 24, 26, 28, 29, 30,
                        34, 35]
Primitive roots of 43: [3, 5, 12, 18, 19, 20, 26, 28, 29, 30, 33, 34]
Primitive roots of 47: [5, 10, 11, 13, 15, 19, 20, 22, 23, 26, 29, 30, 31, 33,
                        35, 38, 39, 40, 41, 43, 44, 45]
\end{verbatim}

\item A aplicação mais comum para raízes primitivas módulo $n$ ocorre no
método assimétrico de troca de chaves chamado de Diffie-Hellman. Neste método, o
número primo base e uma de suas raízes primitivas são utilizados para calcular a
chave secreta final, resultado da aritmética modular proposta pelos autores.
Este método é considerado seguro pois, para que este seja inviabilizado, é
necessário resolver um problema chamado de \emph{logaritmo discreto}: um inteiro
$k$ que resolve a equação $b^k = g$, onde $b$ e $g$ são elementos (não
necessariamente números reais) cuidadosamente escolhidos. Métodos eficientes
para a resolução deste problema não são conhecidos, e diversos algoritmos de
criptografia assimétrica baseiam sua segurança nessa `'dificuldade`'.

Uma aplicação menos conhecida ocorre no design de difusores acústicos modulares
\cite{Walker:1990}; um difusor com uma estrutura baseada em raízes primitivas
possibilita a prevenção de reflexão de onda na direção
especular\footnote{reflexão de ondas similar a um espelho, onde uma onda é
refletida para uma direção única, como num lago.}, assim absorvendo uma maior
quantidade de som, neste caso.

\end{itemize}

\bibliography{ine5429_t2}
\bibliographystyle{plain}

\end{document}
