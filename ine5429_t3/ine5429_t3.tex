\documentclass{article}

\usepackage[brazil]{babel}
\usepackage[T1]{fontenc}
\usepackage[a4paper, margin=1.5cm]{geometry}
\usepackage[colorlinks, urlcolor=blue, citecolor=red]{hyperref}
\usepackage[utf8]{inputenc}
\usepackage{amsfonts, mathtools}

\title{\textbf{Protocolo de Acordo de Chaves de Diffie-Hellman}}
\author{Gustavo Zambonin\thanks{\texttt{gustavo.zambonin@grad.ufsc.br} ---
todos os algoritmos utilizados podem ser encontrados
também \href{https://github.com/zambonin/ufsc-ine5429}{neste repositório}.} \\
\small {Segurança em Computação (UFSC -- INE5429)} \vspace{-5mm}}
\date{}

\begin{document}

\maketitle

\begin{itemize}

\item Originalmente publicado em 1975, o \emph{protocolo de acordo de chaves
Diffie--Hellman} (DH) \cite{Diffie:2006:NDC:2263321.2269104} habilita duas
entidades a estabelecerem uma \emph{chave secreta} mesmo que estas não tenham se
comunicado previamente, com o benefício desta troca de dados poder ser
monitorada sem que tal chave seja descoberta. Esta pode ser utilizada,
possivelmente, em um algoritmo de cifragem simétrico, de modo a habilitar a
troca de mensagens criptografadas entre as entidades. Uma distinção importante é
que \emph{nenhuma} informação é trocada além das inerentes ao processo de
criação de chave -- assim sendo, o protocolo DH \emph{não} é classificado como
um algoritmo de criptografia assimétrica.

Uma descrição matemática pode ser resumida da seguinte maneira: sejam Alice e
Bob as entidades em questão; um número primo $g$ e uma de suas raízes primitivas
$p$, ambos suficientemente grandes (e geralmente publicamente padronizados
\cite{rfc5114}) são escolhidos por Alice e Bob. Cada uma das entidades gera
um número secreto para si, chamados de $X_A$ e $X_B$; Alice calcula $A = g^{X_A}
\pmod{p}$ e envia para Bob; do mesmo modo, $B = g^{X_B} \pmod{p}$ é enviado para
Alice por Bob. Assim, as únicas trocas propostas são efetuadas e as entidades já
compartilham uma chave secreta.

Esta reside no cálculo de $B^{X_A} \pmod{p}$ por Alice e $A^{X_B} \pmod{p}$ por
Bob. Expandindo os números recebidos, temos $(g^{X_A} \pmod{p})^{X_B} \pmod{p}$
e $(g^{X_B} \pmod{p})^{X_A} \pmod{p}$ respectivamente; é possível reduzir estes
números, através de aritmética modular, para $g^{X_A X_B} \pmod{p}$ e $g^{X_B
X_A} \pmod{p}$; nota-se a igualdade dos termos, em virtude da comutatividade da
operação de multiplicação sob os números reais. Assim, Alice nunca soube o
número secreto de Bob e vice-versa, e a chave secreta nunca foi transmitida,
porém é conhecida por ambos.
\begin{align*}
Alice \overset{A}{\longrightarrow} Bob \\
Alice \overset{B}{\longleftarrow} Bob
\end{align*}
Utilizando um exemplo numérico, supõe-se que Alice e Bob concordam em usar o
número primo $p = 1949$ e sua raiz primitiva $g = 1475$. Seus números secretos
são, respectivamente, $X_A = 128$ e $X_B = 64$. O segredo compartilhado é
chamado de $s$.
\begin{align*}
A &= g^{X_A} \pmod{p} = 1475^{128} \pmod{1979} = 448 \\
B &= g^{X_B} \pmod{p} = 1475^{64} \pmod{1979} = 872 \\
s &= B^{X_A} \pmod{p} = 872^{128} \pmod{1979}
   = A^{X_B} \pmod{p} = 448^{64} \pmod{1979} = 560
\end{align*}
Estas computações podem ser realizadas rapidamente com um interpretador Python,
pois a linguagem implementa exponenciação modular\footnote{operação do tipo
$d = a^b \pmod{c}$ onde a exponenciação, um passo extremamente custoso se $a$ e
$b$ forem números relativamente grandes, não precisa ser calculada. O resultado
$d$ é a inversa multiplicativa de $a \pmod{c}$.}.

\begin{verbatim}
>>> pow(1475, 128, 1979)
448
>>> pow(1475, 64, 1979)
872
>>> pow(448, 64, 1979)
560
>>> pow(872, 128, 1979) == pow(448, 64, 1979)
True
\end{verbatim}

\item O programa que implementa DH está localizado em
\texttt{diffie\_hellman.py}, e o acordo de chaves automatizado, em
\texttt{key\_exchange.py}. Este programa apresentará falha (e saída)
\emph{apenas} se a chave secreta não for igual para ambas as entidades,
consequência de algum processo matemático anômalo. É importante notar que os
números e suas raízes primitivas foram computados utilizando algoritmos
implementados pelo autor em trabalhos passados, porém os segredos de cada
entidade foram computados utilizando o gerador de números aleatórios da
linguagem, similar ao implementado.

O tamanho de $\approx 80$ bits obtido para os números primos aleatórios acontece
pois a extração de raízes primitivas é um processo custoso; depois de
otimizações no cálculo da função totiente de Euler e da utilização de uma
implementação alternativa da linguagem
Python\footnote{\href{http://pypy.org/}{PyPy} --- o principal recurso
apresentado é a utilização de um compilador JIT (\emph{just-in-time}).},
percebeu-se que uma abordagem diferente seria necessária para que tal processo
fosse acelerado, como uma mudança de linguagem de programação, e assim todo o
ecossistema já construído para o trabalho precisaria ser refeito. Portanto, a
prova de conceito é demonstrada com números de tamanho reduzido.

\item Um ataque do tipo \emph{man-in-the-middle} existe quando uma entidade
monitora secretamente a comunicação entre duas outras entidades, podendo alterar
as mensagens entre elas, simulando uma impersonificação dupla simultânea; como o
protocolo DH original não contém qualquer tipo de autenticação dos usuários, ele
torna-se vulnerável a este tipo de ataque. Caso o atacante, chamado de Eve,
consiga obter as chaves parciais $A$ e $B$ de Alice e Bob, então é possível que
ele crie uma chave secreta com cada um deles e simule a troca de mensagens
direta caso o canal não seja seguro o suficiente. Sejam as entidades Alice, Bob
e Eve, seus números secretos $X_A = 128$, $X_B = 64$ e $X_E = 32$, $p = 1949$ e
$g = 1475$. O segredo compartilhado entre Alice e Eve é chamado de $s_{ae}$, e
entre Eve e Bob, $s_{eb}$.
\begin{align*}
A &= g^{X_A} \pmod{p} = 1475^{128} \pmod{1979} = 448 \\
E &= g^{X_E} \pmod{p} = 1475^{32} \pmod{1979} = 542 \\
B &= g^{X_B} \pmod{p} = 1475^{64} \pmod{1979} = 872 \\
s_{ae} &= E^{X_A} \pmod{p} = 542^{128} \pmod{1979}
        = A^{X_E} \pmod{p} = 448^{32} \pmod{1979} = 560 \\
s_{eb} &= E^{X_B} \pmod{p} = 542^{64} \pmod{1979}
        = B^{X_E} \pmod{p} = 872^{32} \pmod{1979} = 322
\end{align*}
Um diagrama segue abaixo, mostrando uma mensagem $M$ e uma mensagem alterada por
Eve $M'$ a partir do ataque descrito (decodificada com $s_{ae}$ e codificada com
$s_{eb}$).
\begin{align*}
Alice \overset{A}{\longrightarrow} Eve \overset{A}{\longrightarrow} Bob \\
Alice \overset{E}{\longleftarrow} Eve \overset{B}{\longleftarrow} Bob \\
Alice \overset{M}{\longrightarrow} Eve \overset{M'}{\longrightarrow} Bob
\end{align*}

\item Se $g$ não for uma raiz primitiva módulo $p$, então $g$ gerará apenas um
subgrupo do grupo multiplicativo $\mathbb{Z}/p\mathbb{Z}$, e assim a segurança
do protocolo será proporcional à \emph{ordem} de $g$ em
$\mathbb{Z}/p\mathbb{Z}$, onde o ideal seria a ordem total do grupo. Para que
esse aspecto seja contornado, um número suficientemente grande é escolhido para
que a ordem de seu subgrupo seja consequemente afetada, e assim, um nível de
segurança plausível seja obtido.

\end{itemize}

\bibliography{ine5429_t3}
\bibliographystyle{plain}

\end{document}
