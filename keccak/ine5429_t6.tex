\documentclass{article}

\usepackage[brazil]{babel}
\usepackage[T1]{fontenc}
\usepackage[a4paper, margin=1.5cm]{geometry}
\usepackage[colorlinks, urlcolor=blue, citecolor=red]{hyperref}
\usepackage[utf8]{inputenc}
\usepackage{amsfonts, enumitem}

\title{\textbf{A função \emph{hash} criptográfica SHA-3}}
\author{Gustavo Zambonin\thanks{\texttt{gustavo.zambonin@grad.ufsc.br} ---
todos os algoritmos utilizados podem ser encontrados
também \href{https://github.com/zambonin/ufsc-ine5429}{neste repositório}.} \\
\small {Segurança em Computação (UFSC -- INE5429)} \vspace{-5mm}}
\date{}

\begin{document}

\maketitle

\begin{itemize}

\item Uma \emph{função hash criptográfica}, ou função de resumo criptográfica 
(futuramente denotada por $h$), é um algoritmo matemático que mapeia uma
quantidade de bytes qualquer\footnote{algumas funções desse tipo têm
limites quanto ao tamanho da entrada, embora estes sejam extremamente
grandes.} para uma palavra de tamanho fixo, ou seja,
$h : \{0, 1\}^{*} \longrightarrow \{0, 1\}^{n}$, $n \in \mathbb{N}$.

Para que seja resistente a diversos tipos de criptoanálise, uma função
$h : X \longrightarrow Y$ deve respeitar algumas propriedades:

\begin{enumerate}[label=\roman*.]

\item \emph{Resistência à pré-imagem}: Para um resumo $M' \in Y$, é
computacionalmente impraticável\footnote{o tempo ou recursos gastos para esta
computação excedem a validade ou utilidade da informação desejada.} encontrar a
mensagem $M \in X$ tal que $h(M) = M'$. Uma função matemática com esta
propriedade é chamada de unidirecional.

\item \emph{Resistência à segunda pré-imagem}: Para uma mensagem $M_0 \in X$,
é computacionalmente impraticável encontrar uma segunda mensagem $M_1 \in X$
tal que $M_0 \neq M_1$ e $h(M_0) = h(M_1)$.

\item \emph{Resistência à colisão}: Para duas mensagens $M_0, M_1 \in X$, é
computacionalmente impraticável encontrar $M_0 \neq M_1$ e $h(M_0) = h(M_1)$.

\end{enumerate}

É importante notar que, embora as definições sejam extremamente parecidas,
resistência à segunda pré-imagem e resistência à colisão são conceitos
diferentes; um atacante não consegue escolher a primeira mensagem caso queira
atacar a resistência à segunda pré-imagem; para a resistência à colisão, o
atacante pode escolher livremente o par de mensagens.

Algumas aplicações destas funções são enumeradas abaixo:

\begin{itemize}

\item Podem ser utilizadas para verificar a integridade da mensagem, comparando
resumos criptográficos calculados antes e depois da transmissão de mensagem e/ou 
arquivos.

\item Para evitar o armazenamento de senhas em texto claro, é possível
armazenar apenas o resumo criptográfico de cada senha e compará-lo na
autenticação do usuário.

\item Resumos criptográficos são comumente descritos como identificadores únicos
seguros para um arquivo ou informação digital (por exemplo, \emph{commits} em um
sistema de controle de versão).

\end{itemize}

O padrão SHA-3, descrito pelo documento FIPS 202 \cite{Dworkin2015},
é baseado em uma instância do algoritmo \textsc{Keccak}, selecionado
pelo NIST (\emph{National Institute of Standards and Technology}).
Este documento também especifica a família \textsc{Keccak}$-p$ de
permutações matemáticas.

\end{itemize}

\bibliography{ine5429_t6}
\bibliographystyle{plain}

\end{document}
