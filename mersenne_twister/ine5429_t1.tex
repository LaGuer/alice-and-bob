\documentclass{article}

\usepackage[brazil]{babel}
\usepackage[T1]{fontenc}
\usepackage[a4paper, margin=1.5cm]{geometry}
\usepackage[colorlinks, urlcolor=blue, citecolor=red]{hyperref}
\usepackage[utf8]{inputenc}
\usepackage{amsfonts, mathtools}

\title{\textbf{Geração de números primos}}
\author{Gustavo Zambonin\thanks{\texttt{gustavo.zambonin@grad.ufsc.br} ---
todos os algoritmos utilizados podem ser encontrados
também \href{https://github.com/zambonin/ufsc-ine5429}{neste repositório}.} \\
\small {Segurança em Computação (UFSC -- INE5429)} \vspace{-5mm}}
\date{}

\begin{document}

\maketitle

\begin{itemize}

\item Diversos métodos para geração de números aleatórios estão disponíveis
como alternativas à necessidade do usuário: por exemplo, um gerador pode focar
em desempenho, enquanto outros podem gerar números com uma maior quantidade de
bits a partir de processos matemáticos mais complexos. Um \emph{gerador de
números pseudoaleatórios} (PRNG, também chamado de gerador de bits
determinístico) é um algoritmo que tem como função gerar sequências de números
aproximadamente aleatórios, dependentes apenas de um pequeno conjunto de
valores iniciais, chamados de semente.

É relevante apontar que, embora tais algoritmos sejam em grande parte
complexos no aspecto teórico, ainda são baseados em uma série de
transformações lineares que podem ser relacionadas com a semente inicial,
tornando a saída dos algoritmos previsível e insegura. Assim sendo, um gerador
de números pseudoaleatórios \emph{criptograficamente seguro} (CSPRNG) é muito
mais recomendável para uso em aplicações sensíveis.

O PRNG discutido é o \emph{Mersenne Twister} (MT,
\cite{Matsumoto:1998:MTE:272991.272995}), o mais difundido e presente em
várias linguagens de programação como o gerador padrão de números
pseudoaleatórios. Seu nome é derivado do período\footnote{a quantidade de
números gerados antes da sequência começar a se repetir.} de $2^{19937} - 1$,
um primo de Mersenne\footnote{número primo na forma $2^n - 1, n \in
\mathbb{Z}$.}. Formalmente, o algoritmo é baseado numa relação de recorrência
linear matricial sobre $\mathbb{F}_2$\footnote{o corpo de Galois de dois
elementos, também representado por $GF(2)$.}. Definindo uma série $x_i$
através de uma relação de recorrência simples, obtêm-se números na forma $x_i
A$, onde $A$ é uma matriz com elementos em $\mathbb{F}_2$, assim
`'temperando`' os elementos de modo recorrente. Este passo pode ser facilmente
revertido através de uma transformação linear, e o padrão dos números gerados
revelado com suficiente observação, assim fundamentando o fato de que este
gerador não é criptograficamente seguro.

\item Uma linguagem com números de precisão arbitrária mostra-se útil para que
exista flexibilidade caso exista a necessidade de geração de números muito
grandes. Assim sendo, a linguagem Python foi escolhida, pois além de sua alta
legibilidade e grande número de recursos embutidos, é possível trabalhar com
números de tamanho indefinido, dado poder computacional existente para tal.
Uma possível implementação genérica para o MT está localizada em
\verb!mt19937.py!, e pode ser executada da seguinte maneira (\verb!seed! deve
ser um número inteiro, e os outros parâmetros utilizados são fornecidos pelos
autores do PRNG, podendo divergir dado o tamanho do inteiro que deseja ser
gerado):

\begin{verbatim}
    $ python
    >>> from mt19937 import MT19937
    >>> mt19937_32 = (32, 624, 397, 31, 0x9908b0df, 11, 0xffffffff, 7,
                      0x9d2c5680, 15, 0xefc60000, 18, 1812433253)
    >>> seed = 13104307
    >>> MT19937(seed, *mt19937_32).generate()
    2594696978
\end{verbatim}

\item A utilização de números primos com fins criptográficos é bem conhecida;
um uso bastante comum é na modalidade assimétrica -- chaves RSA são geradas a
partir de números primos com um número de bits suficiente para que sejam
seguras e praticamente inquebráveis. Assim sendo, devem existir testes de
primalidade que sejam razoavelmente simples de modo a facilitar estes
processos e viabilizar a criptografia. Os dois testes probabilísticos
discutidos são o teste de \emph{Fermat} e o teste de \emph{Miller-Rabin}.

O Pequeno Teorema de Fermat diz que, se $p$ é primo e $0 < a < p$, então
\begin{align*}
a^{p-1} \equiv 1 \pmod{p}
\end{align*}
Deseja-se testar se $p$ é primo, então é possível escolher inteiros $a$
aleatórios no intervalo possível e verificar se a congruência é válida. Se
isto acontecer para muitos valores de $a$, então $p$ \emph{provavelmente} é um
primo. De modo contrário, se um inteiro $a$ gera uma incongruência da forma
\begin{align*}a^{p-1} \not\equiv 1 \pmod{p}\end{align*}
então $a$ é uma \emph{testemunha} do fato de que $p$ é composto.

O teste de Miller-Rabin adiciona facetas a este teorema: seja um primo $p$
onde $p > 2$. $p - 1$ é um número par, e pode ser escrito como $2^s d$ ($s,
d$ inteiros e $d$ ímpar). É possível verificar que um número não é primo se
\begin{align*}
a^d \not\equiv 1 \pmod{p} \land
a^{2^{r}d} \not\equiv -1 \pmod{p} \; \forall \; (0 \leq r \leq s-1)
\end{align*}
de modo que $a$ novamente é uma testemunha. Procede-se da mesma maneira,
escolhendo inteiros $a$ aleatoriamente. Se qualquer uma das congruências acima
proceder, então $p$ não é primo. Similarmente, caso o método retorne
`'verdadeiro`', então $p$ provavelmente é primo. Uma demonstração sobre a
origem das congruências acima pode ser encontrada em
\cite{Miller:1976:RHT:1739937.1740086} e em diversos outros artigos
posteriores que simplificam esta prova. Naturalmente, este teste tem uma
maior complexidade e portanto um tempo de execução maior, porém isso é
relevado pela maior precisão, pois independente do número testado, existe
uma probabilidade de no mínimo $\frac{1}{2}$ de que este seja detectado como
composto, o que não acontece no teste de Fermat, onde existem números
mais e menos facilmente detectados.

O código relevante está localizado em \verb!primality_test.py! e pode ser
executado da seguinte maneira:

\begin{verbatim}
    $ python
    >>> from primality_test import fermat, miller_rabin
    >>> n, k = 253559837810710172535057072944137070561, 10
    >>> miller_rabin(n, k)
    True
\end{verbatim}

\item Por fim, o código localizado em \verb!find_primes.py! é um simples
\emph{script} para encontrar alguns primos de até 4000 bits. Sua saída mostra
o tempo necessário para obter tal número, assim como o inteiro em si. É
possível notar que o processo torna-se extremamente demorado com o número de
bits $> 1800$ por conta das várias operações exponenciais e modulares nos
testes de primalidade.

\begin{verbatim}
    $ python find_primes.py
    Time: 0:00:00.009934    Bits: 100
    Number: 1107641301581031329462575872343
    Time: 0:00:00.595348    Bits: 200
    Number: 347025374246034602632061254452315635868455417461108122309671
    ...
\end{verbatim}

\end{itemize}

\bibliography{ine5429_t1}
\bibliographystyle{plain}

\end{document}
