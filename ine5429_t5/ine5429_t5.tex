\documentclass[fleqn]{article}

\usepackage[brazil]{babel}
\usepackage[T1]{fontenc}
\usepackage[a4paper, margin=1.5cm]{geometry}
\usepackage[colorlinks, urlcolor=blue, citecolor=red]{hyperref}
\usepackage[utf8]{inputenc}
\usepackage[table]{xcolor}
\usepackage{amsthm, amsfonts, enumitem, mathtools}

\title{\textbf{Teoria de números e corpos finitos}}
\author{Gustavo Zambonin\thanks{\texttt{gustavo.zambonin@grad.ufsc.br} ---
todos os algoritmos utilizados podem ser encontrados
também \href{https://github.com/zambonin/ufsc-ine5429}{neste repositório}.} \\
\small {Segurança em Computação (UFSC -- INE5429)} \vspace{-5mm}}
\date{}

\begin{document}

\maketitle

\begin{enumerate}[label=\textbf{\arabic*})]

\item

\begin{enumerate}

\item O algoritmo de Euclides é um método eficiente para calcular o máximo
divisor comum, ou $mdc$, de dois números inteiros. Baseia-se no princípio de
que $mdc(a, b) = mdc(b, a \pmod{b})$.

\begin{proof}
$a, b, q, r \in \mathbb{Z}$, $a \lor b \neq 0$ e $a = qb + r$
(Teorema da Divisão).

$mdc(a, b) \mid a \land mdc(a, b) \mid b
\implies mdc(a, b) \mid (a - qb) = mdc(a, b) \mid r
\implies mdc(a, b) \leq mdc(b, r)$.

$mdc(b, r) \mid b \land mdc(b, r) \mid r
\implies mdc(b, r) \mid (qb + r) = mdc(b, r) \mid a
\implies mdc(b, r) \leq mdc(a, b)$.

$mdc(a, b) \leq mdc(b, r) \land mdc(b, r) \leq mdc(a, b)
\implies mdc(a, b) = mdc(b, r)$.
\end{proof}
É possível verificar que repetir essa computação diminuirá o número de maior
ordem rapidamente, e o procedimento continuará equivalente, tomando o menor
número e o resto da divisão anterior. Seja $a = 2147483647$ e $b = 541$:
\begin{align*}
2147483647 &= 541 \cdot 3969470 + 377 \\
541 &= 377 \cdot 1 + 164 \\
377 &= 164 \cdot 2 + 49 \\
164 &= 49 \cdot 3 + 17 \\
49 &= 17 \cdot 2 + 15 \\
17 &= 15 \cdot 1 + 2 \\
15 &= 2 \cdot 7 + 1 \\
2 &= 1 \cdot 2 + 0 \\
mdc(2147483647, 541) &= \boldsymbol{1}
\end{align*}

\item De modo similar, o algoritmo de Euclides estendido computa todos os
componentes desconhecidos da identidade de Bézout, ou seja, $x, y$ e
$mdc(a, b)$ em $ax + by = mdc(a, b)$. A inversa multiplicativa modular pode
ser descoberta através da retro-substituição.
\begin{align*}
1 &= 2 - 1 \\
1 &= 2 - (15 - 7 \cdot 2) \\
1 &= 8 \cdot 2 - 15 \\
1 &= 8 \cdot (17 - 15) - 15 \\
1 &= 8 \cdot 17 - 9 \cdot 15 \\
1 &= 8 \cdot 17 - 9 \cdot (49 - 17 \cdot 2) \\
1 &= 26 \cdot 17 - 9 \cdot 49 \\
1 &= 26 \cdot (164 - 49 \cdot 3) - 9 \cdot 49 \\
1 &= 26 \cdot 164 - 87 \cdot 49 \\
1 &= 26 \cdot 164 - 87 \cdot (377 - 164 \cdot 2) \\
1 &= 200 \cdot 164 - 87 \cdot 377 \\
1 &= 200 \cdot (541 - 377) - 87 \cdot 377 \\
1 &= 200 \cdot 541 - 287 \cdot 377 \\
1 &= 200 \cdot 541 - 287 \cdot (2147483647 - 3969470 \cdot 541) \\
1 &= 1139238090 \cdot 541 - 287 \cdot 2147483647 \\
\boldsymbol{1139638090} &= 541^{-1} \pmod{2147483647} \\
\boldsymbol{257} &= 2147483647^{-1} \pmod{541}
\end{align*}

\end{enumerate}

\newpage

\item

\begin{enumerate}

\item Um grupo $G$ é um conjunto finito (ou infinito) de elementos equipados
com uma operação binária\footnote{operação que combina dois elementos de um
conjunto não-vazio $S$ de modo a produzir, unicamente, outro elemento
$xy \in S \; \forall x, y \in S$.}. Juntos, devem satisfazer algumas
propriedades fundamentais:

\begin{enumerate}

\item associatividade: $\forall x, y, z \in G, (x y) z = x (y z)$.
\item elemento de identidade: $\exists I \in G : Ix = xI = x,
\forall x \in G$.
\item elemento inverso: $\exists x^{-1} \in G : x x^{-1} = x^{-1} x = I,
\forall x \in G$.

\end{enumerate}

Um exemplo simples de grupo é o conjunto dos inteiros $\mathbb{Z}$ sobre a
operação usual de adição, onde o elemento de identidade é chamado de zero, e
os inversos são representados com um sinal negativo à frente do elemento. Um
grupo onde sua operação binária é comutativa
($\forall x, y \in G, x + y = y + x$) é chamado de grupo abeliano.

\item Um anel $R$ é um conjunto de elementos equipados com duas operações
binárias $(+, \cdot)$, geralmente interpretadas como adição e multiplicação,
respectivamente. $R$ é um grupo abeliano sobre a operação de adição, e
satisfaz também as seguintes propriedades:

\begin{enumerate}

\item distributividade da multiplicação sobre adição à esquerda e à direita:
$\forall x, y, z \in R, x \cdot (y + z) = (x \cdot y) + (x \cdot z)
\land (y + z) \cdot x = (y \cdot x) + (z \cdot x)$.
\item associatividade da multiplicação\footnote{não necessariamente requerida,
mas extremamente utilizada.}: $\forall x, y, z \in R, (x \cdot y) \cdot z
= x \cdot (y \cdot z)$.
\item elemento identidade da multiplicação\footnote{alguns autores definem
anéis sem esta propriedade.}: $\exists I \in R : I \cdot x = x \cdot I = x,
\forall x \in R$.

\end{enumerate}

Um dos anéis mais conhecidos é, novamente, o conjunto dos inteiros
$\mathbb{Z}$. Seus elementos identidade são 0 e 1 para adição e multiplicação,
respectivamente. Este anel, e muitos outros, são comutativos sobre a operação
de multiplicação ($\forall x, y \in R, (x \cdot y) = (y \cdot x)$).

\item Um corpo é um anel não-trivial\footnote{o anel trivial contém apenas um
elemento: a identidade aditiva, que também é multiplicativa, neste caso.}
cujos elementos formam um grupo abeliano sobre a operação de multiplicação.
Então, um corpo satisfaz vários axiomas (associatividade, comutatividade,
distributividade, elemento identidade e elemento inverso, para adição e
multiplicação) e emula apropriadamente as noções de adição, subtração,
multiplicação e divisão. Um corpo de Galois, ou corpo finito, é um corpo que
contém um número finito de elementos, como o conjunto das classes de
congruência de inteiros módulo $n$, onde $n$ é primo, denotado
$\mathbb{Z}/n\mathbb{Z}$.

\end{enumerate}

\item

\begin{enumerate}

\item Um corpo primo é um corpo que não contém subcorpos próprios\footnote{um
subcorpo é estritamente menor, ou seja, de menor cardinalidade, que o corpo
onde está contido.}. O conjunto dos números racionais com as operações usuais
de adição e multiplicação $(\mathbb{Q}, +, \cdot)$ forma um corpo
primo\footnote{\url{
https://proofwiki.org/wiki/Rational_Numbers_form_Prime_Field}}.

\item Corpos finitos de ordem $2^m, m \geq 1$ são chamados de corpos binários.
Os elementos de $GF(2^m)$ são geralmente polinômios cujos coeficientes são 0
ou 1, com grau máximo de $m - 1$. Estes corpos são particularmente adequados
para utilização em computadores, pois suas operações podem ser simuladas por
deslocamentos de bits e portas lógicas \texttt{XOR}. O corpo $GF(2^3)$ contém
os seguintes polinômios:
$\{0, 1, x, x + 1, x^2, x^2 + 1, x^2 + x, x^2 + x + 1\}$.

\end{enumerate}

\item

\begin{enumerate}

\item Um polinômio irredutível é um polinômio não-constante que não consegue
ser fatorado para o produto de outros dois polinômios não-constantes. Esta
propriedade depende do corpo ou anel que os polinômios pertencem.

\item A adição é realizada entre termos do mesmo grau. Porém, já que os
polinômios pertencem ao corpo finito $GF(2^m)$, os coeficientes devem pertencer
às classes de congruência módulo $2$. A multiplicação utiliza a propriedade
usual da distributividade, e o resultado final deve ser reduzido para o grau
máximo $m - 1$ com uma operação de divisão pelo polinômio primitivo do corpo
finito, além da redução de coeficientes.

\item Tomando o polinômio primitivo $x^8 + x^4 + x^3 + x + 1$, deseja- se
resolver $(x^7 + x^5 + x^4 + x^2 + x) \cdot (x^6 + x^4 + x + 1)$ sobre
$GF(2^8)$.
\begin{multline*}
(x^7 + x^5 + x^4 + x^2 + x) \cdot (x^6 + x^4 + x + 1) = \\
= x^{13} + 2x^{11} + x^{10} + x^9 + 3x^8 + 2x^7
+ 2x^6 + 3x^5 + x^4 + x^3 + 2x^2 + x \\
= x^{13} + x^{10} + x^9 + x^8 + x^5 + x^4 + x^3 + x
\pmod{x^8 + x^4 + x^3 + x + 1}
= \boldsymbol{x^5 + x^4 + x^2 + x}
\end{multline*}

\end{enumerate}

\item

\begin{enumerate}

\item $9x \equiv 8 \pmod{7} \\
= 9x \equiv 1 \pmod{7} \\
= \boldsymbol{4 + 7n}, n \in \mathbb{Z}$

\item $x \equiv 5 \pmod{3} \\
= x \equiv 2 \pmod{3} \\
= \boldsymbol{2 + 3n}, n \in \mathbb{Z}$

\item $x \equiv 5 \pmod{-3} \\
= x \equiv -4 \pmod{-3} \\
= \boldsymbol{-1}$

\item $x \equiv -5 \pmod{3} \\
= x \equiv 1 \pmod{3} \\
= \boldsymbol{1 + 3n}, n \in \mathh{Z}$

\item $x \equiv -5 \pmod{-3} \\
= \boldsymbol{-2}$

\item $x \equiv 1234^{-1} \pmod{4321}$

\begin{tabular}{*{2}{c}}
$\begin{aligned}
4321 &= 1234 \cdot 3 + 619 \\
1234 &= 619 \cdot 1 + 615 \\
619 &= 615 \cdot 1 + 4 \\
615 &= 4 \cdot 153 + 3 \\
4 &= 3 \cdot 1 + 1 \\
3 &= 1 \cdot 3 + 0 \\
mdc(4321, 1234) &= \boldsymbol{1}
\end{aligned}$ &
$\begin{aligned}
1 &= 4 - 3 \\
1 &= 4 - (615 - 4 \cdot 153) \\
1 &= 4 \cdot 154 - 615 \\
1 &= (619 - 615) \cdot 154 - 615 \\
1 &= 154 \cdot 619 - 155 \cdot 615 \\
1 &= 154 \cdot 619 - 155 \cdot (1234 - 619) \\
1 &= 309 \cdot 619 - 155 \cdot 1234 \\
1 &= 309 \cdot (4321 - 3 \cdot 1234) - 155 \cdot 1234 \\
1 &= 309 \cdot 4321 - 1082 \cdot 1234 \\
x &= 4321 - 1082 = \boldsymbol{3239}
\end{aligned}$
\end{tabular}

\item $x \equiv -24140 \pmod{40902} \\
= x \equiv 16762 \pmod{40902} \\
= \boldsymbol{16762 + 40902n}, n \in \mathbb{Z}$

\end{enumerate}

\item Tabela multiplicativa para inteiros em $\mathbb{Z}_{11}$, com as
inversas dos elementos destacadas:

\begin{tabular}{>{\color[gray]{0.35}}c *{10}{>{\color[gray]{0.8}}c}}
$\cdot$ & \color[gray]{0.35}{1} & \color[gray]{0.35}{2} &
\color[gray]{0.35}{3} & \color[gray]{0.35}{4} & \color[gray]{0.35}{5} &
\color[gray]{0.35}{6} & \color[gray]{0.35}{7} & \color[gray]{0.35}{8} &
\color[gray]{0.35}{9} & \color[gray]{0.35}{10} \\
1  & \color{black}{1} & 2 & 3 & 4 & 5 & 6 & 7 & 8 & 9 & 10 \\
2  & 2 & 4 & 6 & 8 & 10 & \color{black}{1} & 3 & 5 & 7 & 9 \\
3  & 3 & 6 & 9 & \color{black}{1} & 4 & 7 & 10 & 2 & 5 & 8 \\
4  & 4 & 8 & \color{black}{1} & 5 & 9 & 2 & 6 & 10 & 3 & 7 \\
5  & 5 & 10 & 4 & 9 & 3 & 8 & 2 & 7 & \color{black}{1} & 6 \\
6  & 6 & \color{black}{1} & 7 & 2 & 8 & 3 & 9 & 4 & 10 & 5 \\
7  & 7 & 3 & 10 & 6 & 2 & 9 & 5 & \color{black}{1} & 8 & 4 \\
8  & 8 & 5 & 2 & 10 & 7 & 4 & \color{black}{1} & 9 & 6 & 3 \\
9  & 9 & 7 & 5 & 3 & \color{black}{1} & 10 & 8 & 6 & 4 & 2 \\
10 & 10 & 9 & 8 & 7 & 6 & 5 & 4 & 3 & 2 & \color{black}{1} \\
\end{tabular}

\item

\begin{enumerate}

\item $(7x + 2) - (x^2 + 5)$ em $\mathbb{Z}_{10}[x] \\
= -x^2 + 7x - 3 \\
= \boldsymbol{9x^2 + 7x + 7}$

\item $(6x^2 + x + 3) \cdot (5x^2 + 2)$ em $\mathbb{Z}_{10}[x] \\
= 30x^4 + 5x^3 + 27x^2 + 2x + 6 \\
= \boldsymbol{5x^3 + 7x^2 + 2x + 6}$

\end{enumerate}

\item

\begin{enumerate}

\item Verifica-se que $x^3 + x + 1$ é um polinômio irredutível sobre $GF(2)$
\cite{Stallings:2002:CNS:599893}, então $mdc(x^3 + x + 1, x^2 + x + 1) =
\boldsymbol{1}$.

\item O polinômio $x^2 + 1$ é irredutível sobre $GF(3)$. Portanto,
$mdc(x^3 - x + 1, x^2 + 1) = \boldsymbol{1}$.

\item Todos os coeficientes são reduzidos módulo 101, e inversas
multiplicativas são utilizadas onde necessário.
\begin{align*}
x^5 + 88x^4 + 73x^3 + 83x^2 + 51x + 67 &= (x^2 - 9x + 906) \cdot
(x^3 + 97x^2 + 40x + 38) + (90x^2 + 8x + 80) \\
x^3 + 97x^2 + 40x + 38 &= 55x + 22 \cdot (90x^2 + 8x + 80) + (9x + 96) \\
90x^2 + 8x + 80 &= (10x + 85) \cdot (9x + 96) + 0
\end{align*}
$mdc(x^5 + 88x^4 + 73x^3 + 83x^2 + 51x + 67, x^3 + 97x^2 + 40x + 38)$ sobre
$GF(101) = \boldsymbol{9x + 96}$.

\end{enumerate}

\newpage

\item Tabela aditiva para $GF(2^4), P = x^4 + x + 1$.

\begin{tabular}{c *{16}{>{\color[gray]{0.35}}c}}
\textbf{+} & \color{black}{0} & \color{black}{1} & \color{black}{2}
& \color{black}{3} & \color{black}{4} & \color{black}{5} & \color{black}{6}
& \color{black}{7} & \color{black}{8} & \color{black}{9} & \color{black}{A}
& \color{black}{B} & \color{black}{C} & \color{black}{D} & \color{black}{E}
& \color{black}{F} \\
0 & 0 & 1 & 2 & 3 & 4 & 5 & 6 & 7 & 8 & 9 & A & B & C & D & E & F \\
1 & 1 & 0 & 3 & 2 & 5 & 4 & 7 & 6 & 9 & 8 & B & A & D & C & F & E \\
2 & 2 & 3 & 0 & 1 & 6 & 7 & 4 & 5 & A & B & 8 & 9 & E & F & C & D \\
3 & 3 & 2 & 1 & 0 & 7 & 6 & 5 & 4 & B & A & 9 & 8 & F & E & D & C \\
4 & 4 & 5 & 6 & 7 & 0 & 1 & 2 & 3 & C & D & E & F & 8 & 9 & A & B \\
5 & 5 & 4 & 7 & 6 & 1 & 0 & 3 & 2 & D & C & F & E & 9 & 8 & B & A \\
6 & 6 & 7 & 4 & 5 & 2 & 3 & 0 & 1 & E & F & C & D & A & B & 8 & 9 \\
7 & 7 & 6 & 5 & 4 & 3 & 2 & 1 & 0 & F & E & D & C & B & A & 9 & 8 \\
8 & 8 & 9 & A & B & C & D & E & F & 0 & 1 & 2 & 3 & 4 & 5 & 6 & 7 \\
9 & 9 & 8 & B & A & D & C & F & E & 1 & 0 & 3 & 2 & 5 & 4 & 7 & 6 \\
A & A & B & 8 & 9 & E & F & C & D & 2 & 3 & 0 & 1 & 6 & 7 & 4 & 5 \\
B & B & A & 9 & 8 & F & E & D & C & 3 & 2 & 1 & 0 & 7 & 6 & 5 & 4 \\
C & C & D & E & F & 8 & 9 & A & B & 4 & 5 & 6 & 7 & 0 & 1 & 2 & 3 \\
D & D & C & F & E & 9 & 8 & B & A & 5 & 4 & 7 & 6 & 1 & 0 & 3 & 2 \\
E & E & F & C & D & A & B & 8 & 9 & 6 & 7 & 4 & 5 & 2 & 3 & 0 & 1 \\
F & F & E & D & C & B & A & 9 & 8 & 7 & 6 & 5 & 4 & 3 & 2 & 1 & 0 \\
\end{tabular}

Tabela multiplicativa para $GF(2^4), P = x^4 + x + 1$.

\begin{tabular}{c *{16}{>{\color[gray]{0.35}}c}}
$\cdot$ & \color{black}{1} & \color{black}{2} & \color{black}{3}
& \color{black}{4} & \color{black}{5} & \color{black}{6} & \color{black}{7}
& \color{black}{8} & \color{black}{9} & \color{black}{A} & \color{black}{B}
& \color{black}{C} & \color{black}{D} & \color{black}{E} & \color{black}{F} \\
1 & 1 & 2 & 3 & 4 & 5 & 6 & 7 & 8 & 9 & A & B & C & D & E & F \\
2 & 2 & 4 & 6 & 8 & A & C & E & 3 & 1 & 7 & 5 & B & 9 & F & D \\
3 & 3 & 6 & 5 & C & F & A & 9 & B & 8 & D & E & 7 & 4 & 1 & 2 \\
4 & 4 & 8 & C & 3 & 7 & B & F & 6 & 2 & E & A & 5 & 1 & D & 9 \\
5 & 5 & A & F & 7 & 2 & D & 8 & E & B & 4 & 1 & 9 & C & 3 & 6 \\
6 & 6 & C & A & B & D & 7 & 1 & 5 & 3 & 9 & F & E & 8 & 2 & 4 \\
7 & 7 & E & 9 & F & 8 & 1 & 6 & D & A & 3 & 4 & 2 & 5 & C & B \\
8 & 8 & 3 & B & 6 & E & 5 & D & C & 4 & F & 7 & A & 2 & 9 & 1 \\
9 & 9 & 1 & 8 & 2 & B & 3 & A & 4 & D & 5 & C & 6 & F & 7 & E \\
A & A & 7 & D & E & 4 & 9 & 3 & F & 5 & 8 & 2 & 1 & B & 6 & C \\
B & B & 5 & E & A & 1 & F & 4 & 7 & C & 2 & 9 & D & 6 & 8 & 3 \\
C & C & B & 7 & 5 & 9 & E & 2 & A & 6 & 1 & D & F & 3 & 4 & 8 \\
D & D & 9 & 4 & 1 & C & 8 & 5 & 2 & F & B & 6 & 3 & E & A & 7 \\
E & E & F & 1 & D & 3 & 2 & C & 9 & 7 & 6 & 8 & 4 & A & B & 5 \\
F & F & D & 2 & 9 & 6 & 4 & B & 1 & E & C & 3 & 8 & 7 & 5 & A \\
\end{tabular}

\item Considerando a tabela multiplicativa acima, verifica-se que
$(x^3 + x + 1)^{-1} = \boldsymbol{x^2 + 1}$ sobre $GF(2^4)$.

\end{enumerate}

\bibliography{ine5429_t5}
\bibliographystyle{plain}

\end{document}
